%% Generated by Sphinx.
\def\sphinxdocclass{report}
\documentclass[letterpaper,10pt,english]{sphinxmanual}
\ifdefined\pdfpxdimen
   \let\sphinxpxdimen\pdfpxdimen\else\newdimen\sphinxpxdimen
\fi \sphinxpxdimen=.75bp\relax

\usepackage[utf8]{inputenc}
\ifdefined\DeclareUnicodeCharacter
 \ifdefined\DeclareUnicodeCharacterAsOptional
  \DeclareUnicodeCharacter{"00A0}{\nobreakspace}
  \DeclareUnicodeCharacter{"2500}{\sphinxunichar{2500}}
  \DeclareUnicodeCharacter{"2502}{\sphinxunichar{2502}}
  \DeclareUnicodeCharacter{"2514}{\sphinxunichar{2514}}
  \DeclareUnicodeCharacter{"251C}{\sphinxunichar{251C}}
  \DeclareUnicodeCharacter{"2572}{\textbackslash}
 \else
  \DeclareUnicodeCharacter{00A0}{\nobreakspace}
  \DeclareUnicodeCharacter{2500}{\sphinxunichar{2500}}
  \DeclareUnicodeCharacter{2502}{\sphinxunichar{2502}}
  \DeclareUnicodeCharacter{2514}{\sphinxunichar{2514}}
  \DeclareUnicodeCharacter{251C}{\sphinxunichar{251C}}
  \DeclareUnicodeCharacter{2572}{\textbackslash}
 \fi
\fi
\usepackage{cmap}
\usepackage[T1]{fontenc}
\usepackage{amsmath,amssymb,amstext}
\usepackage{babel}
\usepackage{times}
\usepackage[Bjarne]{fncychap}
\usepackage[dontkeepoldnames]{sphinx}

\usepackage{geometry}

% Include hyperref last.
\usepackage{hyperref}
% Fix anchor placement for figures with captions.
\usepackage{hypcap}% it must be loaded after hyperref.
% Set up styles of URL: it should be placed after hyperref.
\urlstyle{same}
\addto\captionsenglish{\renewcommand{\contentsname}{Contents:}}

\addto\captionsenglish{\renewcommand{\figurename}{Fig.}}
\addto\captionsenglish{\renewcommand{\tablename}{Table}}
\addto\captionsenglish{\renewcommand{\literalblockname}{Listing}}

\addto\captionsenglish{\renewcommand{\literalblockcontinuedname}{continued from previous page}}
\addto\captionsenglish{\renewcommand{\literalblockcontinuesname}{continues on next page}}

\addto\extrasenglish{\def\pageautorefname{page}}

\setcounter{tocdepth}{1}



\title{Linear Regression Documentation}
\date{Sep 25, 2018}
\release{0.01}
\author{Tommy Myrvik, Kristian Tuv}
\newcommand{\sphinxlogo}{\vbox{}}
\renewcommand{\releasename}{Release}
\makeindex

\begin{document}

\maketitle
\sphinxtableofcontents
\phantomsection\label{\detokenize{index::doc}}

\phantomsection\label{\detokenize{index:module-cls_reg}}\index{cls\_reg (module)}\index{LinReg (class in cls\_reg)}

\begin{fulllineitems}
\phantomsection\label{\detokenize{index:cls_reg.LinReg}}\pysiglinewithargsret{\sphinxbfcode{class }\sphinxcode{cls\_reg.}\sphinxbfcode{LinReg}}{\emph{x}, \emph{y}, \emph{z}, \emph{deg}}{}~\index{MSE() (cls\_reg.LinReg method)}

\begin{fulllineitems}
\phantomsection\label{\detokenize{index:cls_reg.LinReg.MSE}}\pysiglinewithargsret{\sphinxbfcode{MSE}}{\emph{z}, \emph{zpred}}{}
Finds the mean squared error of the real data and predicted values
:param z: real data
:param zpred: predicted data
:type z: array
:type zpred: array
:return: The mean squared error
:rtype: float

\end{fulllineitems}

\index{R2() (cls\_reg.LinReg method)}

\begin{fulllineitems}
\phantomsection\label{\detokenize{index:cls_reg.LinReg.R2}}\pysiglinewithargsret{\sphinxbfcode{R2}}{\emph{z}, \emph{zpred}}{}
Finds the R2 error of the real data and predicted values
:param z: real data
:param zpred: predicted data
:type z: array
:type zpred: array
:return: The mean squared error
:rtype: float

\end{fulllineitems}

\index{\_\_init\_\_() (cls\_reg.LinReg method)}

\begin{fulllineitems}
\phantomsection\label{\detokenize{index:cls_reg.LinReg.__init__}}\pysiglinewithargsret{\sphinxbfcode{\_\_init\_\_}}{\emph{x}, \emph{y}, \emph{z}, \emph{deg}}{}~\begin{quote}\begin{description}
\item[{Parameters}] \leavevmode\begin{itemize}
\item {} 
\sphinxstyleliteralstrong{XY} (\sphinxstyleliteralemphasis{array}) \textendash{} A matrix of polynomialvalues

\item {} 
\sphinxstyleliteralstrong{z} (\sphinxstyleliteralemphasis{array}) \textendash{} The values we are trying to fit

\item {} 
\sphinxstyleliteralstrong{deg} (\sphinxstyleliteralemphasis{int}) \textendash{} The degree of polynomial we try to fit the data

\end{itemize}

\end{description}\end{quote}

\end{fulllineitems}

\index{\_\_weakref\_\_ (cls\_reg.LinReg attribute)}

\begin{fulllineitems}
\phantomsection\label{\detokenize{index:cls_reg.LinReg.__weakref__}}\pysigline{\sphinxbfcode{\_\_weakref\_\_}}
list of weak references to the object (if defined)

\end{fulllineitems}

\index{bootstrap() (cls\_reg.LinReg method)}

\begin{fulllineitems}
\phantomsection\label{\detokenize{index:cls_reg.LinReg.bootstrap}}\pysiglinewithargsret{\sphinxbfcode{bootstrap}}{\emph{nBoots}}{}
I dont fucking know

\end{fulllineitems}

\index{kfold() (cls\_reg.LinReg method)}

\begin{fulllineitems}
\phantomsection\label{\detokenize{index:cls_reg.LinReg.kfold}}\pysiglinewithargsret{\sphinxbfcode{kfold}}{\emph{nfolds}}{}
I dont fucking know

\end{fulllineitems}

\index{lasso() (cls\_reg.LinReg method)}

\begin{fulllineitems}
\phantomsection\label{\detokenize{index:cls_reg.LinReg.lasso}}\pysiglinewithargsret{\sphinxbfcode{lasso}}{\emph{lamb}, \emph{XY=None}, \emph{z=None}}{}
Performes a Lasso regression linear fit
\begin{quote}\begin{description}
\item[{Parameters}] \leavevmode\begin{itemize}
\item {} 
\sphinxstyleliteralstrong{XY} (\sphinxstyleliteralemphasis{array}) \textendash{} A matrix of polynomialvalues

\item {} 
\sphinxstyleliteralstrong{z} (\sphinxstyleliteralemphasis{array}) \textendash{} The values we are trying to fit

\item {} 
\sphinxstyleliteralstrong{lamb} (\sphinxstyleliteralemphasis{float}\sphinxstyleliteralemphasis{, }\sphinxstyleliteralemphasis{int}) \textendash{} The regularization constant

\end{itemize}

\item[{Returns}] \leavevmode
The coefficient of the fitted polynomial

\item[{Return type}] \leavevmode
array

\end{description}\end{quote}

\end{fulllineitems}

\index{ols() (cls\_reg.LinReg method)}

\begin{fulllineitems}
\phantomsection\label{\detokenize{index:cls_reg.LinReg.ols}}\pysiglinewithargsret{\sphinxbfcode{ols}}{\emph{XY=None}, \emph{z=None}}{}
Performes a Ordinary least squares linear fit
\begin{quote}\begin{description}
\item[{Parameters}] \leavevmode\begin{itemize}
\item {} 
\sphinxstyleliteralstrong{XY} (\sphinxstyleliteralemphasis{array}) \textendash{} A matrix of polynomialvalues

\item {} 
\sphinxstyleliteralstrong{z} (\sphinxstyleliteralemphasis{array}) \textendash{} The values we are trying to fit

\end{itemize}

\item[{Returns}] \leavevmode
The coefficient of the fitted polynomial

\item[{Return type}] \leavevmode
array

\end{description}\end{quote}

\end{fulllineitems}

\index{ridge() (cls\_reg.LinReg method)}

\begin{fulllineitems}
\phantomsection\label{\detokenize{index:cls_reg.LinReg.ridge}}\pysiglinewithargsret{\sphinxbfcode{ridge}}{\emph{lamb}, \emph{XY=None}, \emph{z=None}}{}
Performes a Ridge regression linear fit
\begin{quote}\begin{description}
\item[{Parameters}] \leavevmode\begin{itemize}
\item {} 
\sphinxstyleliteralstrong{XY} (\sphinxstyleliteralemphasis{array}) \textendash{} A matrix of polynomialvalues

\item {} 
\sphinxstyleliteralstrong{z} (\sphinxstyleliteralemphasis{array}) \textendash{} The values we are trying to fit

\item {} 
\sphinxstyleliteralstrong{lamb} (\sphinxstyleliteralemphasis{float}\sphinxstyleliteralemphasis{, }\sphinxstyleliteralemphasis{int}) \textendash{} The regularization constant

\end{itemize}

\item[{Returns}] \leavevmode
The coefficient of the fitted polynomial

\item[{Return type}] \leavevmode
array

\end{description}\end{quote}

\end{fulllineitems}


\end{fulllineitems}



\chapter{Indices and tables}
\label{\detokenize{index:indices-and-tables}}\label{\detokenize{index:welcome-to-linear-regression-s-documentation}}\begin{itemize}
\item {} 
\DUrole{xref,std,std-ref}{genindex}

\item {} 
\DUrole{xref,std,std-ref}{modindex}

\item {} 
\DUrole{xref,std,std-ref}{search}

\end{itemize}


\renewcommand{\indexname}{Python Module Index}
\begin{sphinxtheindex}
\def\bigletter#1{{\Large\sffamily#1}\nopagebreak\vspace{1mm}}
\bigletter{c}
\item {\sphinxstyleindexentry{cls\_reg}}\sphinxstyleindexpageref{index:\detokenize{module-cls_reg}}
\end{sphinxtheindex}

\renewcommand{\indexname}{Index}
\printindex
\end{document}